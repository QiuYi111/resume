% a mashup of hipstercv, friggeri and twenty cv
% https://www.latextemplates.com/template/twenty-seconds-resumecv
% https://www.latextemplates.com/template/friggeri-resume-cv

\documentclass[lighthipster]{simplehipstercv}
% available options are: darkhipster, lighthipster, pastel, allblack, grey, verylight, withoutsidebar
% withoutsidebar
%\usepackage[utf8]{inputenc} % XeLaTeX 不需要这个
%\usepackage[default]{raleway} % 如果需要 Raleway 字体,保留
\usepackage[margin=1cm, a4paper]{geometry}

% 引入 ctex 宏包
\usepackage{ctex}

% 设置中文字体 (根据你的系统和喜好选择)
% 方案一:使用系统自带字体 (Windows)
%\setCJKmainfont{SimSun}       % 宋体
%\setCJKsansfont{SimHei}      % 黑体
%\setCJKmonofont{FangSong}    % 仿宋

% 方案二:使用 Noto Sans CJK (推荐,跨平台,覆盖广)
\setCJKmainfont{Noto Sans CJK SC}
\setCJKsansfont{Noto Sans CJK SC}
\setCJKmonofont{Noto Sans CJK SC}

% 方案三:使用思源宋体和思源黑体 (Adobe 和 Google 合作开发)
%\setCJKmainfont{Source Han Serif SC}
%\setCJKsansfont{Source Han Sans SC}
%\setCJKmonofont{Source Han Sans SC}


%------------------------------------------------------------------ Variablen

\newlength{\rightcolwidth}
\newlength{\leftcolwidth}
\setlength{\leftcolwidth}{0.23\textwidth}
\setlength{\rightcolwidth}{0.75\textwidth}

%------------------------------------------------------------------
\title{新简历} % 中文标题
\author{LaTeX 忍者} % 中文作者名
\date{2019年6月} % 中文日期

\pagestyle{empty}
\begin{document}


\thispagestyle{empty}
%-------------------------------------------------------------

\section*{开始} % 中文 section

\simpleheader{headercolour}{邱璟祎}{}{}{white} % 中文名字



%------------------------------------------------

% this has to be here so the paracols starts..
\subsection*{}
\vspace{4em}

\setlength{\columnsep}{1.5cm}
\columnratio{0.23}[0.75]
\begin{paracol}{2}
\hbadness5000
%\backgroundcolor{c[1]}[rgb]{1,1,0.8} % cream yellow for column-1 %\backgroundcolor{g}[rgb]{0.8,1,1} % \backgroundcolor{l}[rgb]{0,0,0.7} % dark blue for left margin

\paracolbackgroundoptions

% 0.9,0.9,0.9 -- 0.8,0.8,0.8


\footnotesize
{\setasidefontcolour
\flushright
\begin{center}
    \roundpic{Cv.jpg}
\end{center}


\bigskip

\bg{cvgreen}{white}{个人信息} \\[0.5em] % 中文
邱璟祎 % 中文

云南·曲靖 % 中文

2003

\bigskip

\bg{cvgreen}{white}{专业} \\[0.5em] % 中文

机械工程 % 中文

\bigskip



\bigskip

\bg{cvgreen}{white}{负责项目}\\[0.5em] % 中文
硬件开发
\bigskip

\bg{cvgreen}{white}{工具与知识}\\[0.5em] %中文

\texttt{Fusion360} ~/~ \texttt{Python} ~/~ \texttt{Linux}\\

\texttt{机器人运动控制} ~/~ \texttt{电子设计} ~/~ \texttt{机械设计}


\vspace{4em}
\infobubble{\faAt}{cvgreen}{white}{qiuyi200311@gmail.com}
\infobubble{\faGithub}{cvgreen}{white}{QiuYi111}

\phantom{turn the page}

\phantom{turn the page}
}
%-----------------------------------------------------------
\switchcolumn

\small
\section*{教育经历} % 中文

\begin{tabular}{r| p{0.4\textwidth} c}
    \cvevent{2022.9--2026.6(预计)}{清华大学}{本科}{北京 \color{cvred}}{机械工程系机械工程实验班}{tsinghua.jpeg}\\ % 中文
    \cvevent{2024.8--2025.2}{荷兰代尔夫特理工大学}{交换}{代尔夫特 \color{cvred}}{Mechanical Engineering}{tud.jpeg} % 中文
\end{tabular}
\vspace{3em}

\section*{项目经历} % 中文
\begin{tabular}{r| p{0.5\textwidth} c}
    \cvevent{2024.01--2024.11}{SRT 外骨骼研究项目}{主要负责人}{清华大学机械工程系 \color{cvred}}{项目聚焦于提升外骨骼设备在续航能力与运动灵活性之间的平衡,通过端到端模型技术采集学习真人步态数据,以提升外骨骼设备的灵活性。我在项目中负责广泛阅读前沿学术成果,提出端到端步态仿真和行动预测的创新方案;引入 AutoGPT 等 AI agent 工具,革新科研工作范式,提升科研效率}{tsinghua.jpeg} \\ % 中文
    \cvevent{2024.11 - 2025.2}{双足机器人模仿学习}{强化学习工程师-实习}{桥介数物(深圳) \color{cvred}}{复现领域前沿论文并进行迁移适配,包括 PHC 渐进式模仿学习和 H2O Sim2Real 策略等。我负责阅读并复现领域前沿文献,将前沿工作迁移到公司自有机器人平台,完成迁移适配。}{} \\ % 中文
    \cvevent{2025.7-至今}{双臂遥操作系统}{算法-实习}{京东探索研究院 \color{cvred}}{我负责 XR 遥操作系统的构建。我将采用主流范
式中的局域网通信架构,结合宇树方案和 TacAR 方案,
实现包含手部姿态遥操作、视频串流和视触觉传感器反
馈的遥操作系统。}{} \\ % 中文
    
\end{tabular}




\vspace{3em}








\vfill{} % Whitespace before final footer

%----------------------------------------------------------------------------------------
%	FINAL FOOTER
%----------------------------------------------------------------------------------------
\setlength{\parindent}{0pt}
\begin{minipage}[t]{\rightcolwidth}
\begin{center}\fontfamily{\sfdefault}\selectfont \color{black!70}
{\small 邱璟祎 \icon{\faEnvelopeO}{cvgreen}{} qiuyi200311@gmail.com \icon{\faMapMarker}{cvgreen}{北京}  \icon{\faPhone}{cvgreen}{}  +86 18810986551 
}
\end{center}
\end{minipage}

\end{paracol}

\end{document}


% !TeX root = ../thuthesis-example.tex

\chapter{研究背景及意义}

脑机接口旨在建立生物大脑与外部系统之间的直接信息交互通路,使神经系统能够绕过传统的周围神经与肌肉执行路径,对外部设备或环境施加影响。随着神经科学实验技术与计算方法的持续发展,脑机接口研究正逐步从早期的概念验证与单一任务解码,迈向更高层次的目标解析与工程系统验证阶段。

传统脑机接口研究多聚焦于初级运动皮层中与肢体运动相关的神经活动,通过运动意图等离散变量和解码位置、速度等连续变量实现对运动轨迹的重建。然而,该范式在长期稳定性、用户认知负载以及复杂任务环境中的鲁棒性方面逐渐显现出局限性。近年来,随着对神经系统高维群体表征理解的加深,学术界开始关注直接解码与目标规划、空间记忆等高层认知变量的控制范式,以期在降低底层控制复杂度的同时提升系统整体可用性。

在这一转变背景下,海马体因其在空间认知与目标表征中的关键作用而受到广泛关注。已有神经科学研究表明,海马体神经群体活动能够在一定条件下维持与目标位置相关的稳定表征状态,为目标导向型脑机接口提供了重要的生理基础。与此同时,光学脑机接口技术,尤其是双光子钙成像手段,为在单神经元分辨率下观测大规模神经群体活动提供了可能,使直接解析相关神经流形成为现实。

然而,相较于电生理手段,光学成像在时间分辨率、信号动力学及数据处理延迟等方面存在内在约束,限制了其在高带宽连续控制任务中的直接应用。因此,在光学脑机接口条件下,如何合理界定系统目标层级、构建与信号特性相匹配的解码与具身执行框架,是实现工程可行性的关键问题。

基于上述背景,本文聚焦于在双光子成像条件下,探索基于海马体目标态解码的脑机接口系统设计思路,重点关注目标解码信息在工程系统中的可提取性、稳定性及其与物理执行空间之间的映射一致性,而不追求高精度连续运动控制性能。通过系统性的技术调研、工程预研与系统架构设计,本文旨在为认知解码层级脑机接口的实现提供可行的工程参考。



\chapter{系统设计}

本章围绕构建一套从单神经元尺度神经活动观测到物理空间呈现的闭环脑机接口系统展开,旨在在工程条件下验证认知目标态解码信号的可提取性、时间稳定性及其与物理执行空间之间的映射一致性。
鉴于双光子成像在采样频率及钙离子指示剂动力学特性上的内在限制,本项目在系统设计层面采取“高层目标态调节为主、底层连续控制为辅”的策略,重点关注低频认知目标信息的解算与具身化呈现,而非追求高带宽连续运动轨迹的精确重建。
\section{系统整体设计}

本项目的整体系统设计围绕神经信号动力学特性与物理执行系统响应特性之间的低通动力学匹配这一核心原则展开。双光子钙成像系统的有效采样频率通常约为 30Hz。在此条件下,若直接追求高带宽连续轨迹控制,不仅在工程上难以实现,也不利于动物形成稳定、可解释的认知—行为映射关系。

因此,系统并不以重建连续运动轨迹为目标,而是聚焦于从海马体神经群体活动中提取与空间目标相关的认知目标态表征,并在物理空间中对该目标态进行呈现反馈。

系统整体架构划分为以下三层:

\begin{enumerate}
    \item \textbf{神经表征}:  
    通过双光子显微成像系统获取海马体神经元群体活动,并基于时间一致性与空间位置监督约束的表示学习策略,从高维、稀疏的神经活动空间中提取与空间位置高度相关的低维神经流形表示
    
    \item \textbf{任务逻辑}:  
    在 Bonsai 软件框架下集成实时信号流处理模块,对神经流形的动态演化进行在线处理,并结合 Harp 硬件套件实现奖惩信号、行为输入与神经信号之间的高精度时间同步,为动物学习稳定的目标态偏置提供必要的闭环条件。
    
    \item \textbf{具身执行}:  
    将解算得到的目标态坐标映射至实体机械臂的工作空间。在该架构中,机械臂并非作为高带宽控制对象,而主要作为一种\textbf{物理空间显示与具身化平台},用于将动物的认知目标在三维现实空间中进行直观呈现。
\end{enumerate}

需要强调的是,该系统设计并不以构建高精度、通用型脑机接口为目标,而是用于探索目标态信号在工程条件下的可实现性边界。

\section{解码算法架构设计}

\subsection{数据流水线}
\begin{figure}
    \centering
    \includegraphics[width=0.8\linewidth]{figures/pipeline.png}
    \caption{数据流水线图示}
    \label{fig:my_label}
\end{figure}
本项目构建了一套端到端的数据处理流水线,以支持准实时的神经信号解码与目标态呈现。双光子原始成像数据经由采集模块实时获取后,首先完成运动校正与 ROI 荧光信号提取,形成神经元群体活动向量。
随后,该活动向量被输入至预训练的 CEBRA 表示模型,用于将高维神经活动映射至具备拓扑一致性与时间稳定性的低维潜在表示空间。最终,通过浅层神经网络在该潜在空间中完成目标态读出,输出与动物当前认知目标相关的坐标信息。这种“\textbf{强表示、轻读出}”的解码架构,有助于在保证可解释性的同时降低在线推理的计算复杂度。

\subsection{重点开发组件}

由于双光子成像产生的原始图像数据量远高于后续神经活动向量的规模,系统整体时延主要集中于数据流水线的前端阶段。因此,本项目的工程开发重点包括以下两个核心组件:

\begin{itemize}
    \item \textbf{实时钙信号提取组件}:  
    针对高通量图像数据,需开发实时处理算法,以确保在有限计算资源条件下完成运动校正、荧光信号提取以及对钙信号进行解卷积,提取神经活动信号,避免过长延时。
    
    \item \textbf{浅层网络读出组件}:  
    由于神经流形已通过 CEBRA 完成非线性降维与对齐,目标态读出阶段仅需较为简单的非线性映射即可实现坐标估计,从而著降低在线解码的计算负担。
\end{itemize}

\section{训练范式的构建}

\subsection{硬件基础:行为装置的机电控制与反馈}

本项目采用 Harp 硬件套件构建底层行为控制与反馈系统。Harp 套件通过统一的时钟同步机制,实现了气浮球风扇控制、奖励给水阀门触发以及虚拟现实帧同步信号之间的时间对齐。  
该硬件基础为构建稳定的“感知—反馈”闭环提供了必要条件。

\subsection{任务设计与分阶段塑形}

针对海马体在空间表征与目标相关编码中的功能特性,本项目设计了一套基于虚拟光标导航的训练范式,并采用分阶段塑形策略逐步引导动物形成稳定的目标态。

在初始阶段,小鼠通过转动气浮球直接控制虚拟环境中的光标运动。当光标与蓝色目标物发生接触时,系统触发奖励。为降低学习难度,初期设置多个尺寸较大的目标物。随着训练推进,目标物数量逐渐减少、尺度逐步缩小,从而促使动物建立更加精细的“空间位置—奖励”关联关系,并诱发海马体中稳定的目标相关神经表征。

在动物于虚拟环境中达到稳定表现后,系统进入具身迁移阶段。此时,气浮球对光标的直接控制通路被关闭,取而代之的是由实时解码得到的神经目标态信号对机械臂末端目标位置施加高层引导。动物需通过调节其内部目标态相关的神经活动,使机械臂末端在物理工作空间中逐步逼近对应的蓝色目标物。

\begin{figure}[htbp]
    \centering
    \includegraphics[width=\linewidth]{figures/training.png}
    \caption{基于目标尺度动态演缩与控制句柄迁移的训练逻辑示意图。该范式通过降低初期学习难度,逐步引导动物将物理机械臂内化为其认知空间中的具身呈现终端。}
    \label{fig:shaping_logic}
\end{figure}

\section{具身操作系统的搭建与基本控制}

\subsection{机械臂实验场景设置}

机械臂工作空间与小鼠所在的虚拟现实空间通过坐标变换关系进行对齐。物理空间中的目标位置贴有与虚拟环境一致的蓝色标志物,以在视觉语义层面保持一致性,从而促进跨模态映射关系的形成。

\subsection{机械臂的闭环控制逻辑}

在本系统架构中,机械臂不作为高带宽从动控制对象,而是作为动物认知目标态的\textbf{物理载体与呈现媒介}。具身操作系统负责将解码得到的低频目标态坐标转化为连续、平滑的机械臂末端运动指令,使机械臂在工作空间内呈现出与动物认知目标一致的空间指向。

该“目标态引导”模式在系统层面缓解了钙信号高频信息缺失对连续控制性能的影响,使得在认知目标层面开展稳定的具身交互探索成为可能。


\chapter{脑机接口的范式革新}
\section{介绍}
% 简单做一个帽段介绍脑机接口技术。强调科技、学术上的重要性以及国际科技竞争局势。强调硅谷融资情况和国内政策驱动。
脑机接口(Brain-Machine Interface, BMI)是指在生物大脑与外部设备之间建立的直接信息通信通路。通过实时获取、解码及转换神经信号,该技术允许大脑指令跨越传统的周围神经与肌肉组织,直接驱动外部执行器或计算环境\cite{Wolpaw2002}。

当前,脑机接口技术已跃升为全球科技竞争的关键战略制高点。其研究不仅关乎人类认知的深度解析,更在医疗修复、具身智能及新一代人机交互领域展现出变革性潜力。从学术研究视角看,脑机接口正经历着从“底层运动解析”向“高维认知解码”的范式转移;从产业应用视角看,它正处于从实验室原型向工程化落地的跨越期。

在技术路径的演进中,高通量电极植入(如 Neuralink 方案)与非侵入/微创多模态技术正形成互补之势。以 Merge Labs 为代表的研究机构,在 OpenAI 首席执行官 Sam Altman 的资本支持以及加州理工学院 Mikhail Shapiro 教授的引领下,正利用声学报告基因与磁场信号,探索具备高时空分辨率的超声神经接口。这一路径尝试规避传统电学接口的生物兼容性瓶颈,通过多维度信号获取更多解析维度,代表了国际前沿对脑机融合新机制的积极探索。

在国内,脑机接口已被提升至国家战略高度。在《中共中央关于制定国民经济和社会发展第十五个五年规划的建议》中,脑机接口被明确列为前瞻布局的未来产业。随后启动的第二期中国脑计划——“脑科学与类脑研究”国家重大专项,系统性地部署了脑认知原理解析、脑重大疾病研究以及脑机智能技术应用三大支柱。针对传感器技术的瓶颈,工业和信息化部等七部门于 2024 年 7 月联合发布了《关于推动脑机接口产业创新发展的实施意见》,明确提出要突破传统单一电信号采集模式,重点发展基于光、磁、超声及化学等多模态的新型脑信号传感器。

这一政策导向与学术范式的转向高度契合。传统研究长期聚焦于初级运动皮层(M1)的运动轨迹还原,但随着神经科学研究的深入,学术界意识到实现高阶交互必须跨越底层指令,直接解码大脑深层的目标规划、空间记忆等认知编码。这种复杂的意图通常编码在数以千计、分布稀疏的神经元构成的流形中,使得传统电学手段在空间分辨率与大规模同步观测上的局限性日益凸显。

综上所述,在政策红利驱动、学术范式转型与技术路径迭代的交汇期,脑机接口领域呈现出战略意义深远、技术多元并行、工程挑战巨大的竞争态势。
\subsection{脑机接口的模态}
脑机接口模态系指获取神经活动信号所依赖的物理或化学媒介,其确立了神经解译的信息通量边界与实时性上限。随微纳制造、生物光子学技术及多物理场成像的发展,当前已构建起电生理、光学成像、功能性超声(fUS)及磁共振成像(fMRI)等多维观测方式。这些模态在时空分辨率、探测深度、系统复杂性及生物相容性间呈现出技术权衡:电生理模态以响应精度与信号直接性长期占据主流地位,但缺乏空间分辨率;光学模态通过高特异性的神经元群落观测,突破了传统电极的空间采样瓶颈,但往往受制于时间分辨率;而超声等其他模态在跨脑区全局观测与深部结构探测方面展现出独特价值,但由于其生理机制而受到限制。

\subsubsection{电生理脑机接口}

鉴于神经系统信息传递的电生物学本质,电生理 BCI 通过直接捕获突触后电位或动作电位,系目前主流路径。经半个世纪的科学探究与工程迭代,电生理 BCI 已演进出从宏观场电位到单细胞尺度监测的多尺度观测能力。其技术的演进是在侵入性创伤、时空分辨率与集成规模三者之间不断权衡并推进其边界。根据探测界面与神经组织的解剖关系,本项目拟将主流电生理设备划分为以下三个技术序列进行综述:
\paragraph{非侵入式与半侵入式} 
传统非侵入式脑电图(EEG)通过头皮电极获取局部场电位,虽具备极高的生物安全性与临床普适性,但受限于体积传导效应与颅骨的低通滤波作用,其空间分辨率与信噪比难以满足精细意图的实时解码需求。皮层脑电(ECoG)通过电极阵列部署于硬膜下/外,虽显著提升了信号带宽,但仍面临开颅手术带来的感染风险。近年来,以 Stentrode 为代表的血管内支架电极 \cite{oxley2016} 提供了全新的技术路径。该模态利用导管技术将电极阵列沿颈静脉植入至运动皮层附近的矢状窦内,在规避传统开颅手术创伤的同时,实现了对高保真皮层神经信号的长时程稳定监测,为微创化 BCI 临床转化开辟了新方向。

\paragraph{硬质微电极阵列} 
为获取单神经元量级的峰电位信息,以 犹他阵列(Utah Array) 为代表的硅基硬质微电极阵列 \cite{maynard1997} 成为领域内的技术基准。通过微加工工艺制备的百通道级针状电极阵列,能够穿透皮层表面并直接记录神经元胞体附近的动作电位。此类设备在运动皮层功能映射、瘫痪肢体运动重建及精密语言了解码中展现出较高的性能和可靠性,被认为是“金标准”。然而,硬质材料与柔软脑组织间的模量差距较大,常常导致机械摩擦,引发慢性炎症反应,进而诱导神经胶质增生,导致信号质量随植入时间增加而迅速衰减。

\paragraph{高通量柔性集成系统} 
针对通道数瓶颈与免疫反应挑战,电生理模态正向大规模、高集成度及生物相容方向演进。Neuropixels \cite{jun2017} 利用先进的 CMOS 集成工艺,在毫米级尺度内集成了上千个有源采样位点,实现了对大脑多脑区、跨尺度神经回路的同步、高时空分辨率观测,极大地拓宽了神经科学研究的信息通量。与此同时,以 Neuralink 为代表的柔性集成系统 \cite{musk2019} 结合了高分子聚合物材料与全自动机器人植入技术。通过极细柔性电极丝避开脑表面微血管并降低组织损伤,该技术在确保单神经元记录精度的前提下,将电生理采集的通量推向万通道级,标志着 BCI 系统进入高通量、实时、全脑观测的时代。

\subsubsection{光学脑机接口}

光学 BCI(oBCI)利用近红外光在生物组织中的“光学窗口”特性,通过光子与神经元的相互作用实现信息的捕获与调制。相较于电生理模态,其核心优势在于规避了物理电极对脑组织的侵入性;且得益于基因编码指示剂的丰富性,oBCI 具备天然的细胞类型特异性,甚至能观测树突棘等亚细胞尺度的结构。

\paragraph{双光子功能成像} 
为了观测神经环路,双光子(2P)扫描成像技术已成为高分辨率读取的核心。通过 GCaMP 等钙指示器,系统可实现在强散射组织环境下的深度成像,同步解析大通量神经群体的放电模式。如结合自适应光学(AO)技术 \cite{helmchen2005},通过对波前畸变的补偿,能够在确保近衍射极限空间分辨率的同时,显著提升探测深度与信噪比,为捕捉微小且快速的钙离子流动提供保障。

\paragraph{小型化与介观显微技术} 
针对传统光学设备对实验动物头端固定的物理约束,小型化显微成像技术(如 Mini2P \cite{zong2022})实现了在自由活动状态下的单细胞分辨率观测,拓宽了行为学相关的神经解码能力。在此基础上,以 RUSH3D 为代表的路径,通过超大视野(FOV)与高帧率的并行扫描机制 \cite{fang2024},弥补了微观成像与宏观脑功能区联系的断层,实现了从单一细胞到全皮层尺度回路动态的跨尺度解析。

\paragraph{光遗传学驱动的闭环化} 
光遗传学(Optogenetics)为 oBCI 赋予了主动介入环路功能的能力。通过在特定神经元表达光敏感离子通道(如 ChR2、NpHR),系统能利用精确受控的光脉冲实现对特定环路的兴奋或抑制性调控\cite{deisseroth2015}。这种将“光学读取”与“光学写入”耦合的架构,使得系统能够闭环调控神经网络,实现从被动解码向主动环路控制的跨越,并展现出超越电刺激的空间精确度。



\subsubsection{超声等其他脑机接口模态}
% 这里讲其他模态的脑机接口。注意此处,由于这些模态本质上并不能直接观测到神经活动,而是依靠血流等间接判断,所以xxxx。带过就行

以功能性超声(fUS)与功能性磁共振成像(fMRI)为代表的模态,主要通过监测血流响应间接反映神经元群落的活跃程度。此类模态在跨脑区全局观测与深部结构探测方面展现出独特价值,但其底层生理机制限制了实时反馈的上限。

\paragraph{功能性超声} 
功能性超声成像(fUS)利用高灵敏度平面波超声多普勒技术,实现了对全脑微脉管系统中血流速度变化的实时监测 \cite{mace2011}。相较于 fMRI,fUS 具备更高的时空分辨率(空间可达百微米级,时间可达毫秒级),且设备集成度更高,能实现自由活动动物的深部脑功能监测。近年来,研究已证明基于 fUS 的血流动力学解码可有效提取灵长类动物的运动意图 \cite{norman2021},展示了其在无损/微创 BCI 领域的巨大潜力。然而,该技术仍受限于血液动力学响应的内源性时延,难以支撑极高精度的闭环交互。

\paragraph{功能性磁共振成像} 
功能性磁共振成像(fMRI)基于血氧水平依赖(BOLD)效应,是目前实现无损全脑功能映射的基石技术。通过对全脑体素(Voxel)的并行监测,fMRI 能够以全局视角解析认知过程、情绪调节及大规模神经网络的交互模式。虽然实时 fMRI 技术(rt-fMRI)已被用于神经反馈训练,但由于其高昂的运行成本、笨重的设备以及血氧响应的滞后性,在便携式、实时闭环 BCI 场景中的应用受到了极大限制。

\paragraph{功能近红外光谱技术} 
功能近红外光谱技术(fNIRS)基于神经血管耦合机制,通过监测血氧动力学变化间接反映神经活动。尽管其具备非侵入性优势,但受限于代谢信号的延迟与空间解析度瓶颈,常常仅作为宏观皮层激活状态的验证性辅助手段,不作为精密观测和调控的主体路径\cite{jobsis1977}\cite{naseer2015}。

\section{脑机接口的任务负载与解码范式}

脑机接口的整体效能不仅受限于神经信号采集的带宽,也更关键地取决于系统对神经信息表征层级的利用方式。根据信息在神经系统中所承载的控制抽象度差异,现有脑机接口技术可概括为从运动过程解算向高层意图表达逐步演进。该演进的核心问题在于,如何在闭环控制中兼顾用户的认知负载与系统指令输出的稳定性与鲁棒性。

\paragraph{轨迹解算} 
传统运动型脑机接口主要依赖初级运动皮层(Primary Motor Cortex, M1)的神经电生理或光学信号,通过建立神经元发放活动与肢体运动参数(如位置、速度与加速度)之间的映射关系,实现对连续运动轨迹的实时解码。

该范式具有较高的时间分辨率,适用于精细连续控制任务。然而,其实际应用面临两方面限制:一方面,神经信号的非平稳性使得解码模型参数随时间漂移,导致高维连续解码的长期稳定性不足;另一方面,由于系统仅提供底层运动参数输出,用户需依赖持续的感觉反馈进行实时修正,增加了闭环控制中的认知负载。在复杂或受限环境中,该类缺乏高层目标约束的控制方式往往表现出较低的鲁棒性。

\paragraph{意图决策} 
为提高解码稳定性,研究者提出了基于意图决策的离散解码范式,其典型应用包括意念打字等系统。该类方法通过将连续运动过程离散化为有限的指令状态,利用对特定动作企图的识别实现高准确率的信息输出\cite{Willett2021}。

尽管该范式在符号输出效率和解码可靠性方面具有明显优势,其本质仍属于动作映射框架:用户需通过想象具体肢体动作或运动轨迹间接触发指令。这种间接控制方式在面对高自由度、强环境交互需求的具身操作任务(如机械臂抓取)时,仍然受到指令表达维度受限的制约,难以实现自然、直觉化的空间交互。

\paragraph{目标导向} 
目标导向解码范式尝试从根本上提升脑机接口的控制抽象层级,其核心思想在于将控制意图与具体运动执行过程分离,即由系统解码用户期望到达的目标状态,而非具体运动路径。实现该目标需要关注具备空间表征能力的脑区,例如能够构建内部认知地图的海马体(Hippocampus)。

已有研究表明,海马体中的位置细胞(Place Cells)不仅能够表征个体当前所在位置,还可编码与当前行为相关的目标位置,体现出一定的非局部表征特性\cite{OKeefe1978}\cite{Foster2006}。基于这一特性,本研究拟通过光学成像手段提取海马体活动中的目标相关空间表征,并将其转化为高层目标状态信号,作为具身操作系统逆运动学模块的输入。通过引入“目标规划—机器执行”的共享控制机制,有望在降低用户控制负担的同时,提高复杂任务场景下脑机接口的整体鲁棒性与实用性。

\section{从运动参数解码到目标态解码的范式转移}

\subsection{运动解码范式的瓶颈}
传统脑机接口中的运动解码研究主要聚焦于初级运动皮层,通过建模神经元群体放电率与肌肉动力学参数之间的映射关系,实现对连续运动轨迹的重建。典型方法包括卡尔曼滤波(Kalman Filter)及其一系列扩展模型,这类方法在受控实验条件下已能够较为准确地预测手臂位置、速度等低维运动变量。

然而,该范式在实际应用中面临显著瓶颈。首先,M1 表征依赖于感觉—运动闭环,一旦视觉或本体感觉反馈出现延迟或中断,解码性能往往下降;其次,基于运动参数的控制要求使用者持续关注和调节,这在复杂或长时间任务中会引入较高的认知与注意负荷。依赖M1的运动解码,本质上是将大脑注意力卷入控制器,而不是专注于任务规划。上述限制表明,仅依赖 M1 的底层运动解码难以满足稳定、低负荷的人机交互需求,从而促使研究者开始探索更高层级的神经表征作为控制信号来源。而实现目标解码,较为核心的脑区是海马体。

\subsection{目标解码与海马体}

\subsubsection{海马体与空间认知}
海马体在空间认知中的核心作用最早由 O'Keefe 对位置细胞(Place Cells)的发现所揭示\cite{OKeefe1978}。位置细胞在个体处于特定空间位置时呈现出选择性放电特性,从群体动力学角度构成了一种对外部环境的空间表征,即“认知地图”。随后,内嗅皮层栅格细胞(Grid Cells)的发现进一步补充了这一框架,表明大脑内部存在支持路径积分与向量导航的系统。二者的协同作用使得海马—内嗅系统不仅能够表征当前位置,还能够支持对空间位移与目标位置的内部计算。

\subsubsection{海马体的抽象概念编码}
近年来的研究表明,海马体的表征能力并不局限于物理空间。Constantinescu 等人通过“鸟实验”证明,海马—内嗅系统可以采用类似位置与栅格编码的方式来表征抽象特征空间,如连续变化的形态维度\cite{Constantinescu2016};Aronov 等人的工作进一步展示了在声音频率维度上出现的“位置样”神经元活动\cite{Aronov2017}。这些结果表明,海马体能够将不同维度的连续变量嵌入到一种近似正交化的流形中,从而在一定程度上降低不同概念或目标之间的表征混叠风险。这一特性为从神经信号中提取相对清晰、可区分的目标态坐标提供了重要的理论依据。

\subsubsection{海马体的非局部表征与意图控制}
支持基于海马体目标解码的关键证据在于其非局部表征(Non-local representation)能力。早期研究发现,海马体在静止或休息状态下会自发重现先前经历的路径序列,即所谓的“回放(Replay)”现象\cite{Foster2006};随后,Pfeiffer 等人进一步提出,类似的序列活动也可用于对未来可能路径的预演(Preplay)\cite{Pfeiffer2013},暗示海马体具备超越即时感知位置的预测性编码能力。

这一思路在 2023 年发表于 \textit{Science} 的“绝地小鼠(Jedi Mice)”研究中得到了更直接的验证\cite{Lai2023}。Lai 等人构建了基于海马体空间表征的脑机接口系统,证明啮齿类动物能够在身体静止、缺乏即时感觉反馈的条件下,主动且持续地激活对应远程位置的神经表征,并利用该表征在虚拟环境中完成导航或目标操控任务。尽管该结果目前仍受限于低维目标空间与高度训练条件,但其核心贡献在于揭示了海马体神经群体能够维持稳定的、与当前物理位置解耦的目标态。

\section{目标解码新范式的工程实现挑战}

尽管已有研究在原理层面验证了利用海马体非局部表征实现意图导航的可行性\cite{Lai2023},但要将该类认知目标解码范式进一步转化为可稳定驱动实体机械臂执行复杂操作任务的脑机接口系统,仍需跨越一系列的关键鸿沟。围绕这一转化过程,本项目面临以下三方面核心工程挑战:

\paragraph{从虚拟仿真向物理实体控制}
现有认知脑机接口研究多基于虚拟现实环境开展,而本项目拟引入实体机械臂作为受控对象。其必要性在于:一方面,物理系统不可避免地引入惯性、摩擦、结构振动等动力学因素及硬件通信时延抖动,对解码算法的鲁棒性提出更高要求,从而为算法可用性提供严格验证;另一方面,从具身认知(Embodied Cognition)视角出发,实体机械臂作为受试小鼠认知目标的物理体现,其带来的感觉反馈(如真实三维空间结构与物理交互约束)更有利于诱发稳定且具生物学意义的海马体表征。

\paragraph{光学神经数据处理的时延瓶颈}
海马体目标意图通常编码于由大量神经元构成的动态流形中。在双光子成像条件下,系统将持续产生高通量原始图像数据。现有研究多采用离线分析流程,而脑机接口应用要求在严格受限的时间窗口内完成从图像获取、神经元活动提取到状态推理的计算,以满足闭环控制对实时性的基本需求。如何在保证解码精度的同时显著压缩端到端处理时延,是本项目亟需突破的瓶颈。

\paragraph{认知流形与物理执行器的映射}
海马体所表征的是具备拓扑不变性的抽象“认知地图”,而机械臂控制则依赖于高自由度关节空间中的逆运动学规划与连续轨迹生成。传统线性解码方法难以刻画海马体神经活动所固有的非线性、稀疏性及时空上下文相关特征。如何构建一种具备非线性映射机制,实现从认知目标状态到物理执行轨迹的平滑转换,是实现直觉化、稳定控制的核心理论问题。

\medskip
基于上述挑战,本项目拟围绕双光子成像系统构建、面向流形结构的解码算法设计以及机械臂操作闭环框架搭建等关键环节,系统探索认知目标解码范式在实体执行系统中的工程实现路径,为高层次脑机接口的应用提供参考。

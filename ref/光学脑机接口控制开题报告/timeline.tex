\chapter{时间安排}


\section{第一阶段:系统准备与参数固化(2025年12月 -- 2026年1月)}
\begin{itemize}
    \item 完成双光子显微成像系统的精密光路对齐与成像参数固化。
    \item 调试气浮球行为装置,完成生化棉整流层的安装与流场稳定性实验验证。
    \item 搭建基于 Harp 硬件的实时同步框架,打通双光子采集、VR 渲染与行为输入的通讯链路。
\end{itemize}

\section{第二阶段:行为范式建立与算法预研(2026年2月 -- 2026年3月)}
\begin{itemize}
    \item 启动实验小鼠的头部固定适应性训练及初阶“虚拟导航”塑形任务。
    \item 采集海马体离线神经数据,训练并优化基于 CEBRA 的流形表示模型。
    \item 编写基于 GPU 加速的实时钙信号提取组件,测试离线训练模型在准实时流水线下的解码精度。
\end{itemize}

\section{第三阶段:闭环集成与具身控制实验(2026年4月 -- 2026年5月)}
\begin{itemize}
    \item 实现“控制迁移”,将任务驱动从气浮球转向实时解码的神经信号。
    \item 部署机械臂具身操作系统,进行实体机械臂的目标指向映射实验。
    \item 记录并分析闭环控制中的时延、成功率及神经流形的稳定性,优化 Min-Jerk 轨迹规划算法。
\end{itemize}

\section{第四阶段:数据分析与论文撰写(2026年6月)}
\begin{itemize}
    \item 对比虚拟环境与实体环境下的解码性能差异,探讨具身反馈对海马表征的影响。
    \item 整理实验图表,完成毕业论文的初稿撰写及查重。
    \item 准备毕业论文答辩,总结本项目在光学脑机接口与具身智能领域的创新点。
\end{itemize}
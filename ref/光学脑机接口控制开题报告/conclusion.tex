\chapter{总结}
本文围绕“在光学脑机接口条件下,认知目标态是否具备工程可提取性与物理呈现可行性”这一核心问题,对基于单神经元分辨率双光子成像的脑机接口系统进行了系统性的文献调研、工程预研与整体架构设计。研究并未以构建高性能或通用型脑机接口为目标,而是在明确生物采样特性与物理系统约束的前提下,重点探索目标导向解码范式在工程系统中的可实现边界。

在研究内容上,本文从脑机接口技术的发展脉络出发,系统梳理了由底层连续运动参数解码向高层认知目标解码演进的范式趋势,并分析了海马体在空间认知与目标表征中的关键作用。在此基础上,对双光子钙成像、神经表示与流形学习方法、虚拟现实行为范式以及具身操作系统等关键技术进行了综合调研,明确指出光学脑机接口在时间分辨率、数据通量与实时闭环控制方面所面临的核心工程约束。

基于上述分析,本文提出了一种以“高层认知目标态调节、低层连续控制为辅”为核心原则的系统设计思路,并构建了从神经信号采集、表示学习、目标态读出到物理执行呈现的整体系统架构。该设计在系统层面规避了光学成像条件下难以支撑的高带宽连续轨迹控制问题,将机械臂明确定位为认知目标的具身化反馈,用于验证目标态神经信号在工程条件下的稳定性、可解释性及跨模态映射一致性。相应地,解码算法采用“强表示、轻读出”的结构设计,以降低在线推理复杂度并提升系统整体鲁棒性。

在工程预研方面,本文完成了双光子成像系统与空间认知行为测试装置的搭建与优化,为后续实验研究提供了必要的硬件基础。同时,通过对气浮球装置、均流结构及系统模块化设计的分析与验证,保障了行为输入采集与神经信号记录在实验条件下的稳定性与可扩展性。上述工程预研工作为后续开展认知目标态解码的可行性验证奠定了可靠的工程基础。

需要指出的是,受限于实验周期、动物训练复杂度以及光学成像固有的时间尺度约束,本文的研究工作主要集中于系统设计、工程研发与方法论验证层面。文中所提出的系统架构与解码范式,旨在为认知层级脑机接口的实现提供一种可行的工程路径参考。

总体而言,本文通过工程视角重新审视了光学脑机接口的应用边界,明确了在单神经元分辨率条件下,目标导向解码相较于连续运动解码在系统稳定性与工程可实现性方面所具有的潜在优势。相关分析与设计工作可为后续在更长时间尺度、更复杂任务条件下开展实验研究提供结构化参考,也为认知层级脑机接口系统的工程化探索提供了可复用的设计思路。
% !TeX root = ../thuthesis-example.tex

\begin{denotation}[3.5cm]
  % --- 脑机接口与神经科学相关 ---
  \item[BMI / BCI] 脑机接口 (Brain-Machine Interface / Brain-Computer Interface)
  \item[M1] 初级运动皮层 (Primary Motor Cortex)
  \item[CA1] 海马体 CA1 区 (Cornu Ammonis 1)
  \item[Place Cells] 位置细胞
  \item[Grid Cells] 栅格细胞
  \item[GECI] 遗传编码钙指示剂 (Genetically Encoded Calcium Indicator)
  \item[GCaMP] 一种常用的绿色荧光钙离子指示蛋白
  \item[Optogenetics] 光遗传学
  \item[BOLD] 血氧水平依赖 (Blood Oxygen Level Dependent)
  
  % --- 成像与光学技术 ---
  \item[2P] 双光子 (Two-Photon)
  \item[OBCI] 光学脑机接口 (Optical Brain-Machine Interface)
  \item[FOV] 视野 (Field of View)
  \item[AO] 自适应光学 (Adaptive Optics)
  \item[fUS] 功能性超声成像 (Functional Ultrasound)
  \item[fMRI] 功能性磁共振成像 (Functional Magnetic Resonance Imaging)
  \item[fNIRS] 功能性近红外光谱技术 (Functional Near-Infrared Spectroscopy)
  \item[EEG / ECoG] 脑电图 / 皮层脑电 (Electroencephalography / Electrocorticography)
  
  % --- 算法与计算模型 ---
  \item[CEBRA] 基于对比学习的行为与表示分析算法
  \item[CANDY] 神经动力学对比分析方法 (Contrastive Analysis of Neural Dynamics)
  \item[ROI] 感兴趣区域 (Region of Interest)
  \item[IK] 逆运动学 (Inverse Kinematics)
  \item[RBF] 径向基函数 (Radial Basis Function)
  \item[Min-Jerk] 最小跳跃 (Minimum Jerk) 轨迹规划
  \item[GPU] 图形处理器 (Graphics Processing Unit)
  
  % --- 工程与硬件相关 ---
  \item[VR] 虚拟现实 (Virtual Reality)
  \item[CFD] 计算流体力学 (Computational Fluid Dynamics)
  \item[CMOS] 互补金属氧化物半导体 (Complementary Metal-Oxide-Semiconductor)
  \item[Harp / Bonsai] 本文采用的硬件同步套件与数据流处理软件框架
  
  % --- 数学符号 ---
  \item[$x / \hat{x}$] 行为状态量 / 行为状态估计值
  \item[$K$] 神经群体活动向量
  \item[$z$] 神经流形中的潜在表示 (Latent State)
  \item[$R$] 气浮球半径
  \item[$m$] 气浮球质量
  \item[$\theta$] 气托半顶角
  \item[$h$] 气膜厚度
  \item[$A_{eff}$] 气托有效托举面积
  \item[$\Delta P$] 静压箱压差
  \item[$k$] 安全系数
  \item[$\tau$] 切向力矩
  \item[$\mu$] 流体动力粘度
  \item[$K_{p}$] 多孔介质渗透率
  \item[$\rho$] 气体密度
  \item[$v$] 气流速度向量
\end{denotation}

% 也可以使用 nomencl 宏包,需要在导言区
% \usepackage{nomencl}
% \makenomenclature

% 在这里输出符号说明
% \printnomenclature[3cm]

% 在正文中的任意为都可以标题
% \nomenclature{PI}{聚酰亚胺}
% \nomenclature{MPI}{聚酰亚胺模型化合物,N-苯基邻苯酰亚胺}
% \nomenclature{PBI}{聚苯并咪唑}
% \nomenclature{MPBI}{聚苯并咪唑模型化合物,N-苯基苯并咪唑}
% \nomenclature{PY}{聚吡咙}
% \nomenclature{PMDA-BDA}{均苯四酸二酐与联苯四胺合成的聚吡咙薄膜}
% \nomenclature{MPY}{聚吡咙模型化合物}
% \nomenclature{As-PPT}{聚苯基不对称三嗪}
% \nomenclature{MAsPPT}{聚苯基不对称三嗪单模型化合物,3,5,6-三苯基-1,2,4-三嗪}
% \nomenclature{DMAsPPT}{聚苯基不对称三嗪双模型化合物(水解实验模型化合物)}
% \nomenclature{S-PPT}{聚苯基对称三嗪}
% \nomenclature{MSPPT}{聚苯基对称三嗪模型化合物,2,4,6-三苯基-1,3,5-三嗪}
% \nomenclature{PPQ}{聚苯基喹噁啉}
% \nomenclature{MPPQ}{聚苯基喹噁啉模型化合物,3,4-二苯基苯并二嗪}
% \nomenclature{HMPI}{聚酰亚胺模型化合物的质子化产物}
% \nomenclature{HMPY}{聚吡咙模型化合物的质子化产物}
% \nomenclature{HMPBI}{聚苯并咪唑模型化合物的质子化产物}
% \nomenclature{HMAsPPT}{聚苯基不对称三嗪模型化合物的质子化产物}
% \nomenclature{HMSPPT}{聚苯基对称三嗪模型化合物的质子化产物}
% \nomenclature{HMPPQ}{聚苯基喹噁啉模型化合物的质子化产物}
% \nomenclature{PDT}{热分解温度}
% \nomenclature{HPLC}{高效液相色谱(High Performance Liquid Chromatography)}
% \nomenclature{HPCE}{高效毛细管电泳色谱(High Performance Capillary lectrophoresis)}
% \nomenclature{LC-MS}{液相色谱-质谱联用(Liquid chromatography-Mass Spectrum)}
% \nomenclature{TIC}{总离子浓度(Total Ion Content)}
% \nomenclature{\textit{ab initio}}{基于第一原理的量子化学计算方法,常称从头算法}
% \nomenclature{DFT}{密度泛函理论(Density Functional Theory)}
% \nomenclature{$E_a$}{化学反应的活化能(Activation Energy)}
% \nomenclature{ZPE}{零点振动能(Zero Vibration Energy)}
% \nomenclature{PES}{势能面(Potential Energy Surface)}
% \nomenclature{TS}{过渡态(Transition State)}
% \nomenclature{TST}{过渡态理论(Transition State Theory)}
% \nomenclature{$\increment G^\neq$}{活化自由能(Activation Free Energy)}
% \nomenclature{$\kappa$}{传输系数(Transmission Coefficient)}
% \nomenclature{IRC}{内禀反应坐标(Intrinsic Reaction Coordinates)}
% \nomenclature{$\nu_i$}{虚频(Imaginary Frequency)}
% \nomenclature{ONIOM}{分层算法(Our own N-layered Integrated molecular Orbital and molecular Mechanics)}
% \nomenclature{SCF}{自洽场(Self-Consistent Field)}
% \nomenclature{SCRF}{自洽反应场(Self-Consistent Reaction Field)}

\chapter{关键技术调研}
\section{双光子成像技术}
光学脑机接口实现单神经元尺度的高分辨率读取,依赖于具备穿透能力与高信噪比的显微观测系统。双光子显微镜采用近红外超短脉冲激光进行激发,能够在活体脑组织中稳定获取单神经元分辨率的动态信号,因此被本项目选作核心脑活动记录工具。
\begin{figure}[htbp]
\centering
\includegraphics[width=0.8\textwidth]{figures/2p.png}
\caption{双光子显微成像系统的基本构成与工作流程示意图(改绘自 Grienberger et al, 2022 \cite{Grienberger2022})}
\label{fig:2p_system}
\end{figure}

\subsubsection{双光子激发原理}
双光子激发的物理机制在于荧光分子在飞秒量级时间窗内同时吸收两个近红外光子,从而完成从基态到激发态的跃迁。该非线性激发过程使激发概率与光强平方成正比,从工程实现角度带来了以下关键优势:
\begin{itemize}
\item \textbf{空间选择性:} 荧光激发仅局限于物镜焦点处的极小体积,天然具备三维光学切片能力,无需共聚焦显微镜中的针孔结构即可有效抑制焦外背景信号。
\item \textbf{观测深度:} 采用 800–1100 nm 波段的近红外激光,利用生物体光学窗口,降低了生物组织中的散射与吸收损耗,使系统能够穿透海马体上方皮层,实现数百微米深度范围内的在体成像。
\item \textbf{光损伤抑制:} 由于焦平面外几乎不发生有效激发,能量积累局部化,有利于降低光毒性并延长在体连续观测的稳定时长。
\end{itemize}

\subsubsection{钙成像原理与动力学约束}
钙成像技术利用遗传编码钙指示剂(GECI)将神经元的电信号跨尺度转化为光学信号,但从实时控制视角审视,该过程引入了两个关键的系统性约束:
\begin{itemize}
\item \textbf{卷积低通特性与相位滞后:} 观测到的荧光信号 $F(t)$ 本质上是神经元放电序列 $s(t)$ 与指示剂脉冲响应函数 $h(t)$的连续卷积。由于 GCaMP 等指示剂的结合与解离动力学常数处于百毫秒量级,该卷积过程表现为强烈的低通滤波效应。这不仅削弱了高频放电信息的增益,更造成了显著的内源性相位滞后,直接限制了闭环控制系统的带宽上限与瞬态稳定性。
\item \textbf{状态依赖的非线性饱和:} 钙指示剂的结合动力学遵循 Hill 方程特征,表现出明显的非线性。在高频放电或高背景钙离子浓度下,指示剂的可结合位点趋于饱和,导致荧光响应增益 $dF/dCa$ 衰减。这种灵敏度压缩效应导致神经解码器在任务高负载区间的线性度变差,增加了实时状态估计的残差。
\end{itemize}

综上,双光子钙成像技术在提供单神经元分辨率甚至亚神经元分辨率的同时,也不可避免地引入高维观测数据与低频钙成像约束,对后续实时解码算法与闭环控制架构提出了在时延补偿、特征对齐与系统鲁棒性方面的明确工程要求。


\section{神经编解码}
\subsubsection{神经编解码的定义与数学形式}
神经编解码用于刻画外部刺激、行为状态与神经群体活动之间的映射关系,是理解神经信息表征与构建脑机接口系统的理论基础。在信息论框架下,神经编码过程可形式化为条件概率分布:给定外部刺激或行为状态 $x$,神经群体活动向量 $K$ 的统计特性由 $P(K|x)$ 描述;相应地,神经解码问题则是基于观测到的神经活动 $K$,对潜在状态 $x$ 进行反向推断,即估计 $P(x|K)$ \cite{Mathis2024}。

在神经系统的多层级结构中,信息表征呈现出由低级感觉特征向高级抽象语义逐步演化的特征。以视觉系统为例,神经表征从视网膜和初级视觉皮层中的高维、非线性编码,逐步转化为下颞叶(IT)皮层中更具抽象性和线性可分性的表征形式。这一层级化演化过程表明,尽管单个神经元活动高度复杂,神经群体层面的信息结构往往可被嵌入到低维潜在空间中加以刻画。

在贝叶斯推断框架下,神经解码可被视为后验概率的最大化问题:
\begin{equation}
    \hat{x} = \arg\max_{x} P(x|K) = \arg\max_{x} \frac{P(K|x)P(x)}{P(K)}
\end{equation}
其中,$P(K|x)$ 表示编码模型(似然项),$P(x)$ 表示先验模型(如行为约束或任务结构),而 $P(K)$ 为归一化常数。由于生物神经系统的编码机制通常具有显著的非线性、噪声与个体差异,传统基于显式假设的模型难以全面刻画其统计特性,因此,数据驱动的潜在变量模型逐渐成为神经编解码研究的重要工具。


\begin{figure}
    \centering
    \includegraphics[width=0.8\linewidth]{figures/decoding.png}
    \caption{神经编解码示意图。(改绘自 Mathis et al, 2024\cite{Mathis2024})}
    \label{fig:my_label}
\end{figure}

\subsection{神经流形学习与动力学解码}
基于潜在变量模型获得的低维表示通常被解释为嵌入在高维神经活动空间中的神经流形(Neural Manifold)。从几何角度看,神经流形刻画了在生物物理约束与网络连接结构限制下,神经群体活动可达的低维状态集合。大量研究表明,不同物种、不同脑区及不同任务条件下的神经活动均呈现出稳定的低维几何结构,体现了神经表征的内在约束性与一致性。

例如,在全脑尺度上,基于钙成像记录的线虫(\textit{C. elegans})神经活动可映射为一致的低维流形结构,不同个体间共享相似的动力学轨迹;在小鼠丘脑中,头方向细胞的群体活动在低维空间中形成环状流形,对应动物的头朝向角度;在内嗅皮层,网格细胞的放电模式在低维嵌入中呈现出环面拓扑结构,用以支持二维空间位置的周期性表征;而在海马体 CA1 区,学习后形成的神经流形结构往往与实验环境的几何特征高度对应,反映了空间经验在神经表征层面的几何映射关系\cite{Perich2025}。

这些研究结果表明,神经群体活动并非在高维空间中无结构地变化,而是受限于低维流形结构之上,为神经解码问题引入了明确的几何约束。
\begin{figure}
    \centering
    \includegraphics[width=\linewidth]{figures/manifold.png}
    \caption{线虫、小鼠在不同任务中的神经流形(改绘自 Perich et al 2025 \cite{Perich2025})}
    \label{fig:my_label}
\end{figure}

\subsubsection{潜在变量学习与表示对齐方法}
围绕神经流形的发现,近年来发展出一系列面向神经时间序列数据的潜在表示学习方法。其中,CEBRA(Contrastive Embedding for Behavior and Representation Analysis)是一类基于对比学习的建模方法,旨在同时利用神经活动与辅助变量(如行为、运动学或任务状态)学习具有可解释性的低维潜在表示 \cite{Schneider2023}。

CEBRA 通过设计正负样本对,并最小化嵌入空间中的表示差异函数 $\phi(\cdot,\cdot)$,促使在行为或时间上相近的神经状态在潜在空间中保持邻近关系,同时拉开无关状态之间的距离。该策略使得学习到的低维表示在保持神经活动拓扑结构的同时,与外部行为变量实现对齐,从而有效缓解高维神经信号在长期记录中的表示漂移问题。此类方法为复杂脑区(如海马体)中稳定神经表征的提取与比较提供了统一的算法框架。

\begin{figure}
    \centering
    \includegraphics[width=\linewidth]{figures/latent_embbeding.png}
    \caption{潜在嵌入空间的编解码(改绘自 Mathis et al, 2024\cite{Mathis2024})}
    \label{fig:my_label}
\end{figure}

\subsection{神经流形的动力学视角}
除静态几何结构外,神经流形的时间演化特性同样是理解神经编解码机制的关键。已有研究提出将神经群体活动视为一个低维自主动力系统,其潜在状态 $z$ 的演化可描述为
\[
\dot{z} = f(z, u),
\]
其中 $u$ 表示外部输入或调制项。在运动控制等任务中,该动力系统常表现出稳定的旋转或轨道结构,使得连续的神经状态演化能够自然地产生平滑的行为输出 \cite{Churchland2012}。

进一步研究表明,尽管单个神经元的放电模式可能随时间发生显著漂移,但群体层面的低维流形结构及其动力学特征在较长时间尺度上保持稳定 \cite{Gallego2017}。这种稳定性为长期神经解码与免校准脑机接口系统的构建提供了重要的理论基础。近期提出的 CANDY(Contrastive Analysis of Neural Dynamics)方法在对比学习框架中引入神经常微分方程,对潜在空间的连续动力学进行显式建模 \cite{Zhu2025CANDY},进一步拓展了神经流形分析在时间连续性和动力学预测方面的能力。

\section{空间认知行为测试装置}
个体对外部世界的感知并非被动刺激的响应,而是通过持续的感知—运动闭环实现的主动过程。在自然行为中,个体会根据内在目标动态调整朝向与运动策略,并通过选择、接近及操作物体完成信息整合。已有研究表明,即使外部刺激在物理层面保持一致,不同的行为状态仍可重塑神经元群体的响应模式。因此,若要解析个体与环境交互过程中神经表征的形成机制,实验范式必须避免被动刺激框架,转而支持自然、连续的行为发生。
\subsection{虚拟现实(VR)在动物认知研究中的作用}
虚拟现实技术(Virtual Reality, VR)为实验动物提供了一种可精确控制的闭环感知—运动环境。在该框架下,动物的自主运动实时驱动感官输入的更新,从而在实验条件中重建感知与行为之间的因果耦合关系。相较于传统刺激呈现范式,VR 系统在保持高度可控性与可重复性的同时,引入了任务依赖的行为自由度,使其成为研究空间导航、认知地图形成及决策过程的重要实验平台。

在系统神经科学研究中,VR 技术常与转基因动物模型、病毒介导表达及遗传编码钙指示剂协同使用,构建“行为—成像”一体化实验框架。该策略使研究者能够在动物执行复杂导航任务的同时,以单细胞分辨率追踪特定神经回路的动态活动,为理解空间认知的神经基础提手段。

\subsection{系统构成与典型范式}
典型的啮齿动物 VR 行为系统由若干功能模块协同构成,其核心目标是在引入可控约束的前提下,最大程度保留自然运动相关的感觉反馈:

\begin{enumerate}
\item \textbf{运动跟踪与约束模块}:常采用球形跑步机结构。动物固定于球体上方,其围绕身体轴线的行走与转向被转化为球体旋转参数,并通过高分辨率光学传感器实时解算动物的虚拟位移。
\item \textbf{多模态感官呈现与反馈模块}:包括覆盖主要视野范围的环幕投影或多屏显示系统,并与奖励或惩罚装置同步,用以构建任务相关的感官—反馈映射。
\end{enumerate}

根据实验目标的不同,VR 系统可在行为自由度与信号稳定性之间进行权衡调整。例如,在需要高时空分辨率光学记录的实验中,头部固定(Head-fixed)方案可有效降低运动伪影;而在研究自由探索或自然策略形成时,则可结合小型化显微成像设备实现相对自由的行为记录 \cite{Thurley2017}。

\begin{figure}
    \centering
    \includegraphics[width=0.8\linewidth]{figures/jetball.png}
    \caption{多种VR气浮球装置(改绘自 Thurley et al, 2017 \cite{Thurley2017})}
    \label{fig:my_label}
\end{figure}

\section{基于课程学习的认知导航范式设计}
在双光子成像平台与神经解码算法已具备的前提下,构建可递进学习、可稳定收敛的课程化行为范式,是实现实验小鼠按照预设意图完成机械臂控制任务的关键工程环节。
\subsection{视觉反馈下的复合任务逻辑}
本项目在行为设计上借鉴了 Makino(2023)提出的基于视觉反馈的闭环控制范式 \cite{Makino2023}。在该范式中,实验动物需根据外部视觉指示主动调节自身运动状态,从而将虚拟物体驱动至预设奖赏区域。已有研究表明,涉及视觉—运动多模态对齐的复合任务学习周期通常长达 2–3 个月,其可行性依赖于将整体任务拆解为基础运动控制与奖赏条件反射两个阶段,并在子任务达到稳定熟练后再进入跨模态整合训练,以实现意图的泛化。
\subsection{分阶段塑形}
从计算学习理论与强化学习框架出发,Lee 等(2024)系统论证了分阶段塑形(Staged Shaping)在复杂行为习得中的必要性 \cite{Lee2024},并揭示了其在脑机接口任务中的三项关键机制:

\paragraph{克服奖励分配问题}
在复杂脑机接口控制任务中,奖励信号往往稀疏且存在延迟,直接进行端到端训练难以准确评估神经活动对最终结果的贡献。分阶段训练通过在各阶段引入明确的局部目标函数,将原本的全局优化问题转化为一系列明确的局部搜索过程,从而有效缓解“奖励分配”难题。

\paragraph{搜索空间的约束与热启动}
随着任务复杂度提升,神经策略参数空间的维度呈指数级增长,直接优化易陷入局部最优或长时间零梯度区域。Lee 等指出,先行完成的子任务训练实质上在低维潜空间内学习得到一组具有生理意义的动作基元。这些基元作为先验知识,为复合任务提供良好的参数初始化,约束后续搜索范围并提高收敛效率。

\paragraph{神经表征的解耦与稳定性}
分阶段塑形有助于在不同脑区间形成相对解耦的模块化神经表征,例如运动皮层中与动力学相关的控制模块,以及海马系统中与空间认知相关的表征模块。该模块化结构使系统在遭遇物理环境扰动时,仅需对受影响子模块进行局部调整,而无需整体重构认知导航策略,从而显著提升脑机接口系统的长期稳定性。

\medskip
综上,基于课程学习的分阶段塑形策略不仅符合实验动物的行为学习规律,更在工程层面有效降低了解空间复杂度并提升神经流形的收敛效率,为本项目在有限实验周期内实现“意念驱动机械臂”的目标提供了可靠的范式支撑。


\section{具身操作系统}
具身操作系统是指受试小鼠通过神经信号操纵的、具备物理实体的装置和配套的算法,是连接神经解码输出与物理世界的中间层。针对海马体目标导向的解码范式,该系统需同时承担三项关键职能:其一,实现虚拟认知空间向物理工作空间的映射;其二,协调生物神经信号采样率(约 30~Hz)与控制系统频率(约 1~kHz)之间的时间尺度差异,保障运动连续性;其三,构建稳定、可预测的闭环反馈机制,以促进目标态相关神经表征的形成与巩固。

\subsection{空间坐标映射与受限逆运动学解算}

海马体神经解码结果通常对应于动物在虚拟环境(如气浮球 VR 场景)中的二维空间位置表征。为使机械臂末端运动能够直观反映动物的空间意图,有必要在虚拟空间 $\mathbb{S}{v}$ 与物理工作空间 $\mathbb{S}{p}$ 之间建立几何一致的映射关系。

为降低动物在闭环控制中的认知负载,机械臂末端运动通常被约束于二维水平平面内。针对多自由度串联机械臂,本项目采用基于雅可比矩阵伪逆的逆运动学求解框架。进一步地,通过引入零空间向量的约束优化策略,在保证末端 $XY$ 坐标精确跟踪的同时,提升机械臂构型的稳定性并规避奇异位形,从而获得更平滑、可预测的物理响应。

\subsection{时间尺度匹配与最小跳跃轨迹规划}

受限于钙指示剂动力学特性与扫描机制,双光子成像系统的有效采样率通常为 $30\text{--}60$~Hz,而机械臂伺服控制频率往往需达到 $500\text{--}1000$~Hz。该数量级差异若未加处理,将导致末端运动呈现明显的离散性与轨迹不连续性。

为弥合采样率差异,需在连续解码点之间进行高频轨迹插值。简单的线性插值虽然实现方便,但会在速度与加速度层面引入不连续性,进而诱发机械冲击与控制不稳定。生物运动研究表明,人类及灵长类动物的自然触及行为符合最小跳跃(Minimum Jerk)原则 \cite{Flash1985},即通过最小化位置对时间的五阶导数生成平滑轨迹。

本项目在具身操作系统中引入 Min-Jerk 轨迹规划器,使机械臂运动在动力学特性上更接近生物运动模式。这不仅有助于降低硬件磨损,更重要的是为动物提供时间上连续、空间上可预测的视觉反馈,从而促进海马体目标态吸引子的稳定形成。

\subsection{感知反馈与闭环系统优化}
脑机接口系统的本质在于神经系统与外部执行体之间的闭环耦合。高质量的感知反馈是动物建立“具身感”并实现稳定操控的关键前提。

现有实验范式多依赖单一视觉反馈通道。然而,由于双光子成像不可避免地引入延迟,动物感知到的视觉反馈往往存在时间延迟。为支持高效的课程学习,具身操作系统需与自动奖惩模块深度耦合。通过提升机械臂末端执行效率并结合性能评估,可根据动物的控制精度动态调整奖励。该基于反馈的自适应优化机制,是实现从随机试错向目标导向主动操控过渡的重要工程基础。